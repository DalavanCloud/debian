%% LyX 1.5.1 created this file.  For more info, see http://www.lyx.org/.
%% Do not edit unless you really know what you are doing.
\documentclass[english]{article}
\usepackage[T1]{fontenc}
\usepackage[latin9]{inputenc}
\setcounter{secnumdepth}{2}
\setcounter{tocdepth}{2}

\makeatletter
\usepackage{babel}
\makeatother

\begin{document}
\tableofcontents{}


\title{XRCed Manual}


\date{Roman Rolinsky <genericsoma@gmail.com>}

\maketitle

\section*{Introduction}

XRCed is a resource editor for editing XRC resource files used in
wxWidgets or wxPython. XRC is an XML-based format for describing the
layout of the interface. When a XRC file it being parsed in the application
(using wxXmlResource class), objects corresponding to the interface
elements defined in .xrc file are created and can be immediately used.
Please refer to the corresponding documentation depending on your
language for details. XRC files are actually normal text files. Structure
of an XML file is defined by a hierarchy of \emph{nodes,} each node
is delimited by a pair of \emph{tags} in a similar manner to HTML
format. XRC format uses a special \texttt{object} tag for defining
interface element nodes, which can contain a number of \emph{attribute}
nodes with tag names corresponding to the attribute names, e.g. \texttt{<pos>100,100</pos>}.
Object nodes always have XML attribute {}``class'' defined, e.g.
\texttt{<object class='wxFrame'>}.


\section{Getting started}

XRCed interface contains these components:

\begin{description}
\item [{XML\ tree}] a tree representing the structure of the resource
file.
\item [{Attribute\ panel}] a panel with the information about the currently
selected tree item. Class and object name is shown in the upper part
and a notebook with a number of pages is shown below for editing object's
attributes.
\item [{Pull-down\ menu}] is shown when a used right-clicks inside the
tree control. Is is used for creating new elements and basic operations
such as copy/paste.
\item [{Component\ panel}] is a graphical menu for creating new elements
by clicking on icons or by dragging them on the test window.
\item [{Test\ window}] can be shown for an already existing part of the
resource tree to see the resulting appearance of the interface.
\end{description}
Using XRCed is easy once you understand some basic principles. First
of all, it is using a structure-oriented tree-based editing technique,
versus point-and-click method used in most GUI builders. The tree
is actually almost a direct respresentation of the hierarchy of \emph{object}
nodes in XRC file. More precisely, only the nodes corresponding to
visible interface elements are shown. An example of an object node
which is not an interface element is the sizeritem node. 

XRCed tries hard to make accessible the full information contained
in XRC file, so even the nodes which are not shown in the tree must
be represented. This is done by adding their property pages to the
attribute panel.

Secondly, creating new elements is context-dependent. The new elements
are inserted in the tree relative to the current selection (except
for drag-and-drop mode when the place is determined dynamically).
Sometimes there is an ambiguity on the exact position, because some
elements are \emph{containers} which can have children, but it may
be needed to create the new element as a sibling node instead of a
child. Sibling mode can be forced by holding \textbf{Ctrl} key on
wxGTK/wxMSW or \textbf{Alt} key on wxMac when inserting a new element.
Similarly, a new element can be inserted as the first child of a container
by holding \textbf{Shift} key (default is to append it as the last
child).


\section{Tree editing}

When you create a new interface element it is attached at some point
to the existing XML tree. The root node of the tree is created automatically
and has only one attribute - the encoding. Root is not shown in the
tree control (to save space), so its attributes are visible when no
other elements are selected. 

As explained in the previous section, the new elements can be put
in different locations relative to the parent element. What to do
if the initial location is not correct, or should be changed in the
development process? There two ways to re-arrange elements:

\begin{enumerate}
\item Cut some element, then select the new parent or sibling and paste
at the new location. Modifier keys can be used in the same way as
for creating to alter the default paste behaviour when using the pull-down
command or the tool button (pasting by using the keyboard shortcut
is always using the default way).
\item Use one of the {}``Move'' menu or toolbar commands to swap elements
or change their parent/child configuration. If a move tool button
is disabled, this means that it does not make sense to do that kind
of move due to incompatible parent/child configuration.
\end{enumerate}
When you cut a subtree, all elements are cut to the clipboard and
will be pasted to the new location in the same configuration. On some
platforms the clipboard is shared among different processes, so it
is possible to start a second copy of XRCed and copy parts of one
tree into another.


\subsection*{Hidden nodes}

Normally one tree element corresponds to a single <object> tag. Sometimes
it makes sense to associate contents of some <object> tags to element
attributes related to the parent. A common example is a sizer object,
which uses \textbf{sizeritem} objects, which have a single child (a
control or another sizer). For such simple objects an attribute page
is added to the child element attributes. The side-effect of this
when copying a child from one parent's subtree into a subtree of a
parent of a different class, is that the hidden node attributes must
be discarded. This is done automatically.


\subsection*{Cut/Copy/Paste}

To accelerate your workflow it may be very useful to use keyboard
shortcuts. XRCed supports standard editing shortcuts for cut/copy/paste
commands. Pasting operation depends on the state of additional modifier
keys to select the desired pasting location relative to the active
tree selection, in a way similar to creating new nodes. \textbf{Shift}
modifier can be used with the toolbar Paste button or tree pulldown
menu to activate {}``insert'' mode. To activate sibling mode on
wxGTK/wxMSW, \textbf{Ctrl} or \textbf{Alt} modifier is used depending
on the platform and the way of introducing Paste command:

\begin{tabular}{ccc} & \textbf{wxGTK and wxMSW} & \textbf{wxMac}\\
Tool button & Ctrl & Alt\\
Pulldown menu & Ctrl & Alt\\
Ctrl+V or Control+V shortcut & Alt & Alt\\
\end{tabular} 


\section{Defining attributes}

Attributes are edited using the attribute panel. The upper part has
the class field and, if applicable, the object name field. Lower part
contains a notebook with one or several pages. Normally the first
page contains the main attributes, the second is {}``Look'n'Feel''
for editing visual behaviour attributes such as font, colors, etc.,
the third is {}``Style'' with styles grouped in two columns - \textbf{Generic}
and \textbf{Specific,} then there may be {}``ExStyle'' page for
extra styles and if there is a hidden node a page with its attributes.


\subsection*{Changing object's class}

There are two ways to change the class of an element: by editing {}``class:''
field in the attributes panel or by using {}``Replace With'' pulldown
menu command. XRCed tried to preserve all attributes compatible with
the new class.


\section{Test window}

Test view is created simply by loading a part of the resource file
using XRC library, so it should look the same as in your application
(without the contents and changes performed by the program, of course).
There are elements which cannot be tested individually, but only as
a part of a subtree.

{}``Test'' command (View>Test or {}``radioactive sign'' tool or
F5) creates a view of the selected element or subtree. If a top-level
element (such as wxDialog) is selected, it is normally created as
a top-level window. When the selected element cannot be a top-level
window, a miniframe is created and the test view is put inside. Currenly
shown parent element is made bold in the tree control while the test
window is opened.

It is possible to select an element by clicking on it in the test
window after activating the {}``Locate'' command. While the test
window is open, it is possible to create new elements by dragging
a component icon on the test window. The parent element is highlighted
by a blue box.


\subsection*{Auto-refresh mode}

When auto-refresh is activated, any changes made to the attibutes
or the tree structure of the currently selected test subtree provoke
test window refresh (reload from the updated resource). Sometimes
it may be prefered to freeze the test window while making modifications
by deactivating auto-refresh (from the View menu or using Alt+A keyboard
shortcut). To force a refresh, press Ctrl+R or select View>Refresh
command.


\section{Component plugins}

Each interface element available in XRCed is defined using a component
plugin. This permits easy extension to include new XRC format features
and classes and using custom application-specific components.

A component plugin is actually a (singleton) object of \textbf{Component}
class or one of derived classes, containing the data essential for
constructing the visual interface for editing element attributes and
defining its behaviour with respect to other components. Components
must be registered with the component manager \textbf{Manager} object.


\subsection*{Defining component plugins}

There are two ways to define component plugins.

\begin{itemize}
\item Creating an object of one of component classes in a Python module
(optionally defining a new class if a specific behaviour which cannot
be provided by the standard classes) is needed. XRCed looks in standard
\texttt{plugins} directory (in \texttt{XRCed} installation directory)
and in each of the directories in \texttt{XRCEDPATH} environment variable.
It tries to import all Python files, except with names starting with
underscore ('\_'). It then looks in sub-directories which contain
\texttt{\_\_init\_\_.py} and imports them using {*}-mode (so it's
possible to group plugins by directory without listing too many in
\texttt{XRCEDPATH}).
\item Another option to define a plugin is to use an XML \emph{manifest}
file. This can be used for simple plugins which do not require defining
new component classes or component groups. The structure of this CRX
file is very similar to a flat XRC file having only top-level nodes
defined using \textbf{component} tag instead of \textbf{object} (that's
why it is called CRX and uses .crx extention). This is done so that
CRX files can be edited using XRCed, if you don't want to edit XML
by hand. A special command-line options \texttt{-m} or \texttt{-{}-meta}
should be used when starting XRCed to include support for so-called
\emph{meta-components. }Attributes of meta-components correspond to
the constructor parameters of Component classes (class name must be
one of known component classes). It is possible to include normal
interface elements in the same CRX file, for easy testing of new plugins.
\end{itemize}
Please refer to the wxPyWiki documentation for further information
and examples about component plugins.
\end{document}

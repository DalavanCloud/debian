\section{\class{wxPoint}}\label{wxpoint}

A {\bf wxPoint} is a useful data structure for graphics operations.
It simply contains integer {\it x} and {\it y} members.

See also \helpref{wxRealPoint}{wxrealpoint} for a floating point version.

\wxheading{Derived from}

None

\wxheading{Include files}

<wx/gdicmn.h>

\wxheading{See also}

\helpref{wxRealPoint}{wxrealpoint}

\latexignore{\rtfignore{\wxheading{Members}}}

\membersection{wxPoint::wxPoint}\label{wxpointctor}

\func{}{wxPoint}{\void}

\func{}{wxPoint}{\param{int}{ x}, \param{int}{ y}}

Create a point.

\membersection{wxPoint::x}\label{wxpointx}

\member{int}{x}

x member.

\membersection{wxPoint::y}\label{wxpointy}

\member{int}{ y}

y member.

\membersection{wxPoint::operator $==$}\label{wxpointequal}

\func{bool}{operator $==$}{\param{const wxPoint\& }{pt}}

Equality operator: returns \true if two points are the same.


\membersection{wxPoint::operator $!=$}\label{wxpointnotequal}

\func{bool}{operator $!=$}{\param{const wxPoint\& }{pt}}

Inequality operator: returns \true if two points are different.


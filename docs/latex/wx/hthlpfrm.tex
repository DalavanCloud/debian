%
% automatically generated by HelpGen from
% helpfrm.h at 24/Oct/99 18:03:10
%

\section{\class{wxHtmlHelpDialog}}\label{wxhtmlhelpdialog}

This class is used by \helpref{wxHtmlHelpController}{wxhtmlhelpcontroller} 
to display help.
It is an internal class and should not be used directly - except for the case
when you're writing your own HTML help controller.

\wxheading{Derived from}

\helpref{wxFrame}{wxframe}

\wxheading{Include files}

<wx/html/helpdlg.h>

\latexignore{\rtfignore{\wxheading{Members}}}

\membersection{wxHtmlHelpDialog::wxHtmlHelpDialog}\label{wxhtmlhelpdialogwxhtmlhelpdialog}

\func{}{wxHtmlHelpDialog}{\param{wxHtmlHelpData* }{data = NULL}}

\func{}{wxHtmlHelpDialog}{\param{wxWindow* }{parent}, \param{int }{wxWindowID}, \param{const wxString\& }{title = wxEmptyString}, \param{int }{style = wxHF\_DEFAULT\_STYLE}, \param{wxHtmlHelpData* }{data = NULL}}

Constructor. For the values of {\it style}, please see the documentation for \helpref{wxHtmlHelpController}{wxhtmlhelpcontroller}.

\membersection{wxHtmlHelpDialog::AddToolbarButtons}\label{wxhtmlhelpdialogaddtoolbarbuttons}

\func{virtual void}{AddToolbarButtons}{\param{wxToolBar *}{toolBar}, \param{int }{style}} 

You may override this virtual method to add more buttons to the help window's
toolbar. {\it toolBar} is a pointer to the toolbar and {\it style} is the style
flag as passed to the Create method.

wxToolBar::Realize is called immediately after returning from this function.

\membersection{wxHtmlHelpDialog::Create}\label{wxhtmlhelpdialogcreate}

\func{bool}{Create}{\param{wxWindow* }{parent}, \param{wxWindowID }{id}, \param{const wxString\& }{title = wxEmptyString}, \param{int }{style = wxHF\_DEFAULT\_STYLE}}

Creates the dialog. See \helpref{the constructor}{wxhtmlhelpdialogwxhtmlhelpdialog}
for a description of the parameters.

\membersection{wxHtmlHelpDialog::GetController}\label{wxhtmlhelpdialoggetcontroller}

\constfunc{wxHtmlHelpController* }{GetController}{\void} 

Returns the help controller associated with the dialog.

\membersection{wxHtmlHelpDialog::ReadCustomization}\label{wxhtmlhelpdialogreadcustomization}

\func{void}{ReadCustomization}{\param{wxConfigBase* }{cfg}, \param{const wxString\& }{path = wxEmptyString}}

Reads the user's settings for this dialog see \helpref{wxHtmlHelpController::ReadCustomization}{wxhtmlhelpcontrollerreadcustomization})

\membersection{wxHtmlHelpDialog::SetController}\label{wxhtmlhelpdialogsetcontroller}

\func{void}{SetController}{\param{wxHtmlHelpController* }{contoller}} 

Sets the help controller associated with the dialog.

\membersection{wxHtmlHelpDialog::SetTitleFormat}\label{wxhtmlhelpdialogsettitleformat}

\func{void}{SetTitleFormat}{\param{const wxString\& }{format}}

Sets the dialog's title format. {\it format} must contain exactly one "\%s"
(it will be replaced by the page title).

\membersection{wxHtmlHelpDialog::WriteCustomization}\label{wxhtmlhelpdialogwritecustomization}

\func{void}{WriteCustomization}{\param{wxConfigBase* }{cfg}, \param{const wxString\& }{path = wxEmptyString}}

Saves the user's settings for this dialog (see \helpref{wxHtmlHelpController::WriteCustomization}{wxhtmlhelpcontrollerwritecustomization}).

\section{\class{wxHtmlHelpFrame}}\label{wxhtmlhelpframe}

This class is used by \helpref{wxHtmlHelpController}{wxhtmlhelpcontroller} 
to display help.
It is an internal class and should not be used directly - except for the case
when you're writing your own HTML help controller.

\wxheading{Derived from}

\helpref{wxFrame}{wxframe}

\wxheading{Include files}

<wx/html/helpfrm.h>

\latexignore{\rtfignore{\wxheading{Members}}}

\membersection{wxHtmlHelpFrame::wxHtmlHelpFrame}\label{wxhtmlhelpframewxhtmlhelpframe}

\func{}{wxHtmlHelpFrame}{\param{wxHtmlHelpData* }{data = NULL}}

\func{}{wxHtmlHelpFrame}{\param{wxWindow* }{parent}, \param{int }{wxWindowID}, \param{const wxString\& }{title = wxEmptyString}, \param{int }{style = wxHF\_DEFAULT\_STYLE}, \param{wxHtmlHelpData* }{data = NULL}}

Constructor. For the values of {\it style}, please see the documentation for \helpref{wxHtmlHelpController}{wxhtmlhelpcontroller}.

\membersection{wxHtmlHelpFrame::AddToolbarButtons}\label{wxhtmlhelpframeaddtoolbarbuttons}

\func{virtual void}{AddToolbarButtons}{\param{wxToolBar *}{toolBar}, \param{int }{style}} 

You may override this virtual method to add more buttons to the help window's
toolbar. {\it toolBar} is a pointer to the toolbar and {\it style} is the style
flag as passed to the Create method.

wxToolBar::Realize is called immediately after returning from this function.

\membersection{wxHtmlHelpFrame::Create}\label{wxhtmlhelpframecreate}

\func{bool}{Create}{\param{wxWindow* }{parent}, \param{wxWindowID }{id}, \param{const wxString\& }{title = wxEmptyString}, \param{int }{style = wxHF\_DEFAULT\_STYLE}}

Creates the frame. See \helpref{the constructor}{wxhtmlhelpframewxhtmlhelpframe}
for a description of the parameters.

\membersection{wxHtmlHelpFrame::GetController}\label{wxhtmlhelpframegetcontroller}

\constfunc{wxHtmlHelpController* }{GetController}{\void} 

Returns the help controller associated with the frame.

\membersection{wxHtmlHelpFrame::ReadCustomization}\label{wxhtmlhelpframereadcustomization}

\func{void}{ReadCustomization}{\param{wxConfigBase* }{cfg}, \param{const wxString\& }{path = wxEmptyString}}

Reads the user's settings for this frame see \helpref{wxHtmlHelpController::ReadCustomization}{wxhtmlhelpcontrollerreadcustomization})

\membersection{wxHtmlHelpFrame::SetController}\label{wxhtmlhelpframesetcontroller}

\func{void}{SetController}{\param{wxHtmlHelpController* }{contoller}} 

Sets the help controller associated with the frame.

\membersection{wxHtmlHelpFrame::SetTitleFormat}\label{wxhtmlhelpframesettitleformat}

\func{void}{SetTitleFormat}{\param{const wxString\& }{format}}

Sets the frame's title format. {\it format} must contain exactly one "\%s"
(it will be replaced by the page title).

\membersection{wxHtmlHelpFrame::WriteCustomization}\label{wxhtmlhelpframewritecustomization}

\func{void}{WriteCustomization}{\param{wxConfigBase* }{cfg}, \param{const wxString\& }{path = wxEmptyString}}

Saves the user's settings for this frame (see \helpref{wxHtmlHelpController::WriteCustomization}{wxhtmlhelpcontrollerwritecustomization}).

\section{\class{wxHtmlHelpWindow}}\label{wxhtmlhelpwindow}

This class is used by \helpref{wxHtmlHelpController}{wxhtmlhelpcontroller} 
to display help within a frame or dialog, but you can use it yourself to create an embedded HTML help window.

For example:

\begin{verbatim}
    // m_embeddedHelpWindow is a wxHtmlHelpWindow
    // m_embeddedHtmlHelp is a wxHtmlHelpController

    // Create embedded HTML Help window
    m_embeddedHelpWindow = new wxHtmlHelpWindow;
    m_embeddedHtmlHelp.UseConfig(config, rootPath); // Set your own config object here
    m_embeddedHtmlHelp.SetHelpWindow(m_embeddedHelpWindow);    
    m_embeddedHelpWindow->Create(this,
        wxID_ANY, wxDefaultPosition, GetClientSize(), wxTAB_TRAVERSAL|wxNO_BORDER, wxHF_DEFAULT_STYLE);        
    m_embeddedHtmlHelp.AddBook(wxFileName(_T("doc.zip")));
\end{verbatim}

You should pass the style wxHF\_EMBEDDED to the style parameter of wxHtmlHelpController to allow
the embedded window to be destroyed independently of the help controller.

\wxheading{Derived from}

\helpref{wxWindow}{wxwindow}

\wxheading{Include files}

<wx/html/helpwnd.h>

\latexignore{\rtfignore{\wxheading{Members}}}

\membersection{wxHtmlHelpWindow::wxHtmlHelpWindow}\label{wxhtmlhelpwindowwxhtmlhelpwindow}

\func{}{wxHtmlHelpWindow}{\param{wxHtmlHelpData* }{data = NULL}}

\func{}{wxHtmlHelpWindow}{\param{wxWindow* }{parent}, \param{int }{wxWindowID}, \param{const wxPoint\&}{ pos = wxDefaultPosition}, \param{const wxSize\&}{ pos = wxDefaultSize}, \param{int }{style = wxTAB\_TRAVERSAL|wxTAB\_wxNO\_BORDER}, \param{int }{helpStyle = wxHF\_DEFAULT\_STYLE}, \param{wxHtmlHelpData* }{data = NULL}}

Constructor.

Constructor. For the values of {\it helpStyle}, please see the documentation for \helpref{wxHtmlHelpController}{wxhtmlhelpcontroller}.

\membersection{wxHtmlHelpWindow::Create}\label{wxhtmlhelpwindowcreate}

\func{bool}{Create}{\param{wxWindow* }{parent}, \param{wxWindowID }{id}, \param{const wxPoint\&}{ pos = wxDefaultPosition}, \param{const wxSize\&}{ pos = wxDefaultSize}, \param{int }{style = wxTAB\_TRAVERSAL|wxTAB\_wxNO\_BORDER}, \param{int }{helpStyle = wxHF\_DEFAULT\_STYLE}, \param{wxHtmlHelpData* }{data = NULL}}

Creates the help window. See \helpref{the constructor}{wxhtmlhelpwindowwxhtmlhelpwindow}
for a description of the parameters.

\membersection{wxHtmlHelpWindow::CreateContents}\label{wxhtmlhelpwindowcreatecontents}

\func{void}{CreateContents}{\void}

Creates contents panel. (May take some time.)

Protected.

\membersection{wxHtmlHelpWindow::CreateIndex}\label{wxhtmlhelpwindowcreateindex}

\func{void}{CreateIndex}{\void}

Creates index panel. (May take some time.)

Protected.

\membersection{wxHtmlHelpWindow::CreateSearch}\label{wxhtmlhelpwindowcreatesearch}

\func{void}{CreateSearch}{\void}

Creates search panel.

\membersection{wxHtmlHelpWindow::Display}\label{wxhtmlhelpwindowdisplay}

\func{bool}{Display}{\param{const wxString\& }{x}}

\func{bool}{Display}{\param{const int }{id}}

Displays page x. If not found it will give the user the choice of
searching books.
Looking for the page runs in these steps:

\begin{enumerate}\itemsep=0pt
\item try to locate file named x (if x is for example "doc/howto.htm")
\item try to open starting page of book x
\item try to find x in contents (if x is for example "How To ...")
\item try to find x in index (if x is for example "How To ...")
\end{enumerate}

The second form takes numeric ID as the parameter.
(uses extension to MS format, <param name="ID" value=id>)

\pythonnote{The second form of this method is named DisplayId in
wxPython.}

\membersection{wxHtmlHelpWindow::DisplayContents}\label{wxhtmlhelpwindowdisplaycontents}

\func{bool}{DisplayContents}{\void}

Displays contents panel.

\membersection{wxHtmlHelpWindow::DisplayIndex}\label{wxhtmlhelpwindowdisplayindex}

\func{bool}{DisplayIndex}{\void}

Displays index panel.

\membersection{wxHtmlHelpWindow::GetData}\label{wxhtmlhelpwindowgetdata}

\func{wxHtmlHelpData*}{GetData}{\void}

Returns the wxHtmlHelpData object, which is usually a pointer to the controller's data.

\membersection{wxHtmlHelpWindow::KeywordSearch}\label{wxhtmlhelpwindowkeywordsearch}

\func{bool}{KeywordSearch}{\param{const wxString\& }{keyword}, \param{wxHelpSearchMode }{mode = wxHELP\_SEARCH\_ALL}}

Search for given keyword. Optionally it searches through the index (mode =
wxHELP\_SEARCH\_INDEX), default the content (mode = wxHELP\_SEARCH\_ALL).

\membersection{wxHtmlHelpWindow::ReadCustomization}\label{wxhtmlhelpwindowreadcustomization}

\func{void}{ReadCustomization}{\param{wxConfigBase* }{cfg}, \param{const wxString\& }{path = wxEmptyString}}

Reads the user's settings for this window (see \helpref{wxHtmlHelpController::ReadCustomization}{wxhtmlhelpcontrollerreadcustomization})

\membersection{wxHtmlHelpWindow::RefreshLists}\label{wxhtmlhelpwindowrefreshlists}

\func{void}{RefreshLists}{\void}

Refresh all panels. This is necessary if a new book was added.

Protected.

\membersection{wxHtmlHelpWindow::SetTitleFormat}\label{wxhtmlhelpwindowsettitleformat}

\func{void}{SetTitleFormat}{\param{const wxString\& }{format}}

Sets the frame's title format. {\it format} must contain exactly one "\%s"
(it will be replaced by the page title).

\membersection{wxHtmlHelpWindow::UseConfig}\label{wxhtmlhelpwindowuseconfig}

\func{void}{UseConfig}{\param{wxConfigBase* }{config}, \param{const wxString\& }{rootpath = wxEmptyString}}

Associates a wxConfig object with the help window. It is recommended that you use \helpref{wxHtmlHelpController::UseConfig}{wxhtmlhelpcontrolleruseconfig} instead.

\membersection{wxHtmlHelpWindow::WriteCustomization}\label{wxhtmlhelpwindowwritecustomization}

\func{void}{WriteCustomization}{\param{wxConfigBase* }{cfg}, \param{const wxString\& }{path = wxEmptyString}}

Saves the user's settings for this window(see \helpref{wxHtmlHelpController::WriteCustomization}{wxhtmlhelpcontrollerwritecustomization}).

\membersection{wxHtmlHelpWindow::AddToolbarButtons}\label{wxhtmlhelpwindowaddtoolbarbuttons}

\func{virtual void}{AddToolbarButtons}{\param{wxToolBar *}{toolBar}, \param{int }{style}} 

You may override this virtual method to add more buttons to the help window's
toolbar. {\it toolBar} is a pointer to the toolbar and {\it style} is the style
flag as passed to the Create method.

wxToolBar::Realize is called immediately after returning from this function.

See {\it samples/html/helpview} for an example.

\section{\class{wxHtmlModalHelp}}\label{wxhtmlmodalhelp}

This class uses \helpref{wxHtmlHelpController}{wxhtmlhelpcontroller} 
to display help in a modal dialog. This is useful on platforms such as wxMac
where if you display help from a modal dialog, the help window must itself be a modal
dialog.

Create objects of this class on the stack, for example:

\begin{verbatim}
    // The help can be browsed during the lifetime of this object; when the user quits
    // the help, program execution will continue.
    wxHtmlModalHelp help(parent, wxT("help"), wxT("My topic"));
\end{verbatim}

\wxheading{Derived from}

None

\wxheading{Include files}

<wx/html/helpctrl.h>

\latexignore{\rtfignore{\wxheading{Members}}}

\membersection{wxHtmlModalHelp::wxHtmlModalHelp}\label{wxhtmlmodalhelpctor}

\func{}{wxHtmlModalHelp}{\param{wxWindow* }{parent}, \param{const wxString\& }{helpFile}, \param{const wxString\& }{topic = wxEmptyString},
        \param{int }{style = wxHF\_DEFAULT\_STYLE | wxHF\_DIALOG | wxHF\_MODAL}}

\wxheading{Parameters}

{\it parent} is the parent of the dialog.

{\it helpFile} is the HTML help file to show.

{\it topic} is an optional topic. If this is empty, the help contents will be shown.

{\it style} is a combination of the flags described in the \helpref{wxHtmlHelpController}{wxhtmlhelpcontroller} documentation.



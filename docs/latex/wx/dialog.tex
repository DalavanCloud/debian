%%%%%%%%%%%%%%%%%%%%%%%%%%%%%%%%%%%%%%%%%%%%%%%%%%%%%%%%%%%%%%%%%%%%%%%%%%%%%%%
%% Name:        dialog.tex
%% Purpose:     wxDialog documentation
%% Author:      wxWidgets Team
%% Modified by:
%% Created:
%% RCS-ID:      $Id: dialog.tex 42921 2006-11-01 20:38:35Z VZ $
%% Copyright:   (c) wxWidgets Team
%% License:     wxWindows license
%%%%%%%%%%%%%%%%%%%%%%%%%%%%%%%%%%%%%%%%%%%%%%%%%%%%%%%%%%%%%%%%%%%%%%%%%%%%%%%

\section{\class{wxDialog}}\label{wxdialog}

A dialog box is a window with a title bar and sometimes a system menu, which
can be moved around the screen. It can contain controls and other windows and
is often used to allow the user to make some choice or to answer a question.


\wxheading{Dialog Buttons}

The dialog usually contains either a single button allowing to close the
dialog or two buttons, one accepting the changes and the other one discarding
them (such button, if present, is automatically activated if the user presses
the \texttt{"Esc"} key). By default, buttons with the standard \texttt{wxID\_OK} 
and \texttt{wxID\_CANCEL} identifiers behave as expected. Starting with
wxWidgets 2.7 it is also possible to use a button with a different identifier
instead, see \helpref{SetAffirmativeId}{wxdialogsetaffirmativeid} and 
\helpref{SetEscapeId}{wxdialogsetescapeid}.

Also notice that the \helpref{CreateButtonSizer()}{wxdialogcreatebuttonsizer} 
should be used to create the buttons appropriate for the current platform and
positioned correctly (including their order which is platform-dependent).



\wxheading{Derived from}

\helpref{wxTopLevelWindow}{wxtoplevelwindow}\\
\helpref{wxWindow}{wxwindow}\\
\helpref{wxEvtHandler}{wxevthandler}\\
\helpref{wxObject}{wxobject}

\wxheading{Include files}

<wx/dialog.h>

\wxheading{Modal and modeless dialogs}

There are two kinds of dialog -- {\it modal}\ and {\it modeless}. A modal dialog
blocks program flow and user input on other windows until it is dismissed,
whereas a modeless dialog behaves more like a frame in that program flow
continues, and input in other windows is still possible. To show a modal dialog
you should use the \helpref{ShowModal}{wxdialogshowmodal} method while to show
a dialog modelessly you simply use \helpref{Show}{wxdialogshow}, just as with
frames.

Note that the modal dialog is one of the very few examples of
wxWindow-derived objects which may be created on the stack and not on the heap.
In other words, although this code snippet:

\begin{verbatim}
    void AskUser()
    {
        MyAskDialog *dlg = new MyAskDialog(...);
        if ( dlg->ShowModal() == wxID_OK )
            ...
        //else: dialog was cancelled or some another button pressed

        dlg->Destroy();
    }
\end{verbatim}

works, you can also achieve the same result by using a simpler code fragment
below:

\begin{verbatim}
    void AskUser()
    {
        MyAskDialog dlg(...);
        if ( dlg.ShowModal() == wxID_OK )
            ...

        // no need to call Destroy() here
    }
\end{verbatim}

An application can define a \helpref{wxCloseEvent}{wxcloseevent} handler for
the dialog to respond to system close events.

\wxheading{Window styles}

\twocolwidtha{5cm}
\begin{twocollist}\itemsep=0pt
\twocolitem{\windowstyle{wxCAPTION}}{Puts a caption on the dialog box.}
\twocolitem{\windowstyle{wxDEFAULT\_DIALOG\_STYLE}}{Equivalent to a combination of wxCAPTION, wxCLOSE\_BOX and wxSYSTEM\_MENU (the last one is not used under Unix)}
\twocolitem{\windowstyle{wxRESIZE\_BORDER}}{Display a resizeable frame around the window.}
\twocolitem{\windowstyle{wxSYSTEM\_MENU}}{Display a system menu.}
\twocolitem{\windowstyle{wxCLOSE\_BOX}}{Displays a close box on the frame.}
\twocolitem{\windowstyle{wxMAXIMIZE\_BOX}}{Displays a maximize box on the dialog.}
\twocolitem{\windowstyle{wxMINIMIZE\_BOX}}{Displays a minimize box on the dialog.}
\twocolitem{\windowstyle{wxTHICK\_FRAME}}{Display a thick frame around the window.}
\twocolitem{\windowstyle{wxSTAY\_ON\_TOP}}{The dialog stays on top of all other windows.}
\twocolitem{\windowstyle{wxNO\_3D}}{Under Windows, specifies that the child controls
should not have 3D borders unless specified in the control.}
\twocolitem{\windowstyle{wxDIALOG\_NO\_PARENT}}{By default, a dialog created
with a {\tt NULL} parent window will be given the
\helpref{application's top level window}{wxappgettopwindow} as parent. Use this
style to prevent this from happening and create an orphan dialog. This is not recommended for modal dialogs.}
\twocolitem{\windowstyle{wxDIALOG\_EX\_CONTEXTHELP}}{Under Windows, puts a query button on the
caption. When pressed, Windows will go into a context-sensitive help mode and wxWidgets will send
a wxEVT\_HELP event if the user clicked on an application window. {\it Note}\ that this is an extended
style and must be set by calling \helpref{SetExtraStyle}{wxwindowsetextrastyle} before Create is called (two-step construction).}
\twocolitem{\windowstyle{wxDIALOG\_EX\_METAL}}{On Mac OS X, frames with this style will be shown with a metallic look. This is an {\it extra} style.}
\end{twocollist}

Under Unix or Linux, MWM (the Motif Window Manager) or other window managers
recognizing the MHM hints should be running for any of these styles to have an
effect.

See also \helpref{Generic window styles}{windowstyles}.

\wxheading{See also}

\helpref{wxDialog overview}{wxdialogoverview}, \helpref{wxFrame}{wxframe},\rtfsp
\helpref{Validator overview}{validatoroverview}

\latexignore{\rtfignore{\wxheading{Members}}}


\membersection{wxDialog::wxDialog}\label{wxdialogctor}

\func{}{wxDialog}{\void}

Default constructor.

\func{}{wxDialog}{\param{wxWindow* }{parent}, \param{wxWindowID }{id},\rtfsp
\param{const wxString\& }{title},\rtfsp
\param{const wxPoint\& }{pos = wxDefaultPosition},\rtfsp
\param{const wxSize\& }{size = wxDefaultSize},\rtfsp
\param{long}{ style = wxDEFAULT\_DIALOG\_STYLE},\rtfsp
\param{const wxString\& }{name = ``dialogBox"}}

Constructor.

\wxheading{Parameters}

\docparam{parent}{Can be NULL, a frame or another dialog box.}

\docparam{id}{An identifier for the dialog. A value of -1 is taken to mean a default.}

\docparam{title}{The title of the dialog.}

\docparam{pos}{The dialog position. A value of (-1, -1) indicates a default position, chosen by
either the windowing system or wxWidgets, depending on platform.}

\docparam{size}{The dialog size. A value of (-1, -1) indicates a default size, chosen by
either the windowing system or wxWidgets, depending on platform.}

\docparam{style}{The window style. See \helpref{wxDialog}{wxdialog}.}

\docparam{name}{Used to associate a name with the window,
allowing the application user to set Motif resource values for
individual dialog boxes.}

\wxheading{See also}

\helpref{wxDialog::Create}{wxdialogcreate}


\membersection{wxDialog::\destruct{wxDialog}}\label{wxdialogdtor}

\func{}{\destruct{wxDialog}}{\void}

Destructor. Deletes any child windows before deleting the physical window.


\membersection{wxDialog::Centre}\label{wxdialogcentre}

\func{void}{Centre}{\param{int}{ direction = wxBOTH}}

Centres the dialog box on the display.

\wxheading{Parameters}

\docparam{direction}{May be {\tt wxHORIZONTAL}, {\tt wxVERTICAL} or {\tt wxBOTH}.}


\membersection{wxDialog::Create}\label{wxdialogcreate}

\func{bool}{Create}{\param{wxWindow* }{parent}, \param{wxWindowID }{id},\rtfsp
\param{const wxString\& }{title},\rtfsp
\param{const wxPoint\& }{pos = wxDefaultPosition},\rtfsp
\param{const wxSize\& }{size = wxDefaultSize},\rtfsp
\param{long}{ style = wxDEFAULT\_DIALOG\_STYLE},\rtfsp
\param{const wxString\& }{name = ``dialogBox"}}

Used for two-step dialog box construction. See \helpref{wxDialog::wxDialog}{wxdialogctor}\rtfsp
for details.


\membersection{wxDialog::CreateButtonSizer}\label{wxdialogcreatebuttonsizer}

\func{wxSizer*}{CreateButtonSizer}{\param{long}{ flags}}

Creates a sizer with standard buttons. {\it flags} is a bit list
of the following flags: wxOK, wxCANCEL, wxYES, wxNO, wxHELP, wxNO\_DEFAULT.

The sizer lays out the buttons in a manner appropriate to the platform.

This function uses \helpref{CreateStdDialogButtonSizer}{wxdialogcreatestddialogbuttonsizer} 
internally for most platforms but doesn't create the sizer at all for the
platforms with hardware buttons (such as smartphones) for which it sets up the
hardware buttons appropriately and returns \NULL, so don't forget to test that
the return value is valid before using it.


\membersection{wxDialog::CreateSeparatedButtonSizer}\label{wxdialogcreateseparatedbuttonsizer}

\func{wxSizer*}{CreateSeparatedButtonSizer}{\param{long}{ flags}}

Creates a sizer with standard buttons using 
\helpref{CreateButtonSizer}{wxdialogcreatebuttonsizer} separated from the rest
of the dialog contents by a horizontal \helpref{wxStaticLine}{wxstaticline}.

Please notice that just like CreateButtonSizer() this function may return \NULL 
if no buttons were created.


\membersection{wxDialog::CreateStdDialogButtonSizer}\label{wxdialogcreatestddialogbuttonsizer}

\func{wxStdDialogButtonSizer*}{CreateStdDialogButtonSizer}{\param{long}{ flags}}

Creates a \helpref{wxStdDialogButtonSizer}{wxstddialogbuttonsizer} with standard buttons. {\it flags} is a bit list
of the following flags: wxOK, wxCANCEL, wxYES, wxNO, wxHELP, wxNO\_DEFAULT.

The sizer lays out the buttons in a manner appropriate to the platform.


\membersection{wxDialog::DoOK}\label{wxdialogdook}

\func{virtual bool}{DoOK}{\void}

This function is called when the titlebar OK button is pressed (PocketPC only).
A command event for the identifier returned by GetAffirmativeId is sent by
default. You can override this function. If the function returns false, wxWidgets
will call Close() for the dialog.


\membersection{wxDialog::EndModal}\label{wxdialogendmodal}

\func{void}{EndModal}{\param{int }{retCode}}

Ends a modal dialog, passing a value to be returned from the \helpref{wxDialog::ShowModal}{wxdialogshowmodal}\rtfsp
invocation.

\wxheading{Parameters}

\docparam{retCode}{The value that should be returned by {\bf ShowModal}.}

\wxheading{See also}

\helpref{wxDialog::ShowModal}{wxdialogshowmodal},\rtfsp
\helpref{wxDialog::GetReturnCode}{wxdialoggetreturncode},\rtfsp
\helpref{wxDialog::SetReturnCode}{wxdialogsetreturncode}


\membersection{wxDialog::GetAffirmativeId}\label{wxdialoggetaffirmativeid}

\constfunc{int}{GetAffirmativeId}{\void}

Gets the identifier of the button which works like standard OK button in this
dialog.

\wxheading{See also}

\helpref{wxDialog::SetAffirmativeId}{wxdialogsetaffirmativeid}


\membersection{wxDialog::GetEscapeId}\label{wxdialoggetescapeid}

\constfunc{int}{GetEscapeId}{\void}

Gets the identifier of the button to map presses of \texttt{\textsc{ESC}}
button to.

\wxheading{See also}

\helpref{wxDialog::SetEscapeId}{wxdialogsetescapeid}


\membersection{wxDialog::GetReturnCode}\label{wxdialoggetreturncode}

\func{int}{GetReturnCode}{\void}

Gets the return code for this window.

\wxheading{Remarks}

A return code is normally associated with a modal dialog, where \helpref{wxDialog::ShowModal}{wxdialogshowmodal} returns
a code to the application.

\wxheading{See also}

\helpref{wxDialog::SetReturnCode}{wxdialogsetreturncode}, \helpref{wxDialog::ShowModal}{wxdialogshowmodal},\rtfsp
\helpref{wxDialog::EndModal}{wxdialogendmodal}


\membersection{wxDialog::GetToolBar}\label{wxdialoggettoolbar}

\constfunc{wxToolBar*}{GetToolBar}{\void}

On PocketPC, a dialog is automatically provided with an empty toolbar. GetToolBar
allows you to access the toolbar and add tools to it. Removing tools and adding
arbitrary controls are not currently supported.

This function is not available on any other platform.


\membersection{wxDialog::Iconize}\label{wxdialogiconized}

\func{void}{Iconize}{\param{const bool}{ iconize}}

Iconizes or restores the dialog. Windows only.

\wxheading{Parameters}

\docparam{iconize}{If true, iconizes the dialog box; if false, shows and restores it.}

\wxheading{Remarks}

Note that in Windows, iconization has no effect since dialog boxes cannot be
iconized. However, applications may need to explicitly restore dialog
boxes under Motif which have user-iconizable frames, and under Windows
calling {\tt Iconize(false)} will bring the window to the front, as does
\rtfsp{\tt Show(true)}.


\membersection{wxDialog::IsIconized}\label{wxdialogisiconized}

\constfunc{bool}{IsIconized}{\void}

Returns true if the dialog box is iconized. Windows only.

\wxheading{Remarks}

Always returns false under Windows since dialogs cannot be iconized.


\membersection{wxDialog::IsModal}\label{wxdialogismodal}

\constfunc{bool}{IsModal}{\void}

Returns true if the dialog box is modal, false otherwise.



\membersection{wxDialog::OnSysColourChanged}\label{wxdialogonsyscolourchanged}

\func{void}{OnSysColourChanged}{\param{wxSysColourChangedEvent\& }{event}}

The default handler for wxEVT\_SYS\_COLOUR\_CHANGED.

\wxheading{Parameters}

\docparam{event}{The colour change event.}

\wxheading{Remarks}

Changes the dialog's colour to conform to the current settings (Windows only).
Add an event table entry for your dialog class if you wish the behaviour
to be different (such as keeping a user-defined
background colour). If you do override this function, call wxEvent::Skip to
propagate the notification to child windows and controls.

\wxheading{See also}

\helpref{wxSysColourChangedEvent}{wxsyscolourchangedevent}


\membersection{wxDialog::SetAffirmativeId}\label{wxdialogsetaffirmativeid}

\func{void}{SetAffirmativeId}{\param{int }{id}}

Sets the identifier to be used as OK button. When the button with this
identifier is pressed, the dialog calls \helpref{Validate}{wxwindowvalidate} 
and \helpref{wxWindow::TransferDataFromWindow}{wxwindowtransferdatafromwindow} 
and, if they both return \true, closes the dialog with \texttt{wxID\_OK} return
code.

Also, when the user presses a hardware OK button on the devices having one or
the special OK button in the PocketPC title bar, an event with this id is
generated.

By default, the affirmative id is wxID\_OK.

\wxheading{See also}

\helpref{wxDialog::GetAffirmativeId}{wxdialoggetaffirmativeid}, \helpref{wxDialog::SetEscapeId}{wxdialogsetescapeid}


\membersection{wxDialog::SetEscapeId}\label{wxdialogsetescapeid}

\func{void}{SetEscapeId}{\param{int }{id}}

Sets the identifier of the button which should work like the standard 
\texttt{\textsc{Cancel}} button in this dialog. When the button with this id is
clicked, the dialog is closed. Also, when the user presses \texttt{\textsc{ESC}} 
key in the dialog or closes the dialog using the close button in the title bar,
this is mapped to the click of the button with the specified id.

By default, the escape id is the special value \texttt{wxID\_ANY} meaning that 
\texttt{wxID\_CANCEL} button is used if it's present in the dialog and
otherwise the button with \helpref{GetAffirmativeId()}{wxdialoggetaffirmativeid} 
is used. Another special value for \arg{id} is \texttt{wxID\_NONE} meaning that
\texttt{\textsc{ESC}} presses should be ignored. If any other value is given, it
is interpreted as the id of the button to map the escape key to.


\membersection{wxDialog::SetIcon}\label{wxdialogseticon}

\func{void}{SetIcon}{\param{const wxIcon\& }{icon}}

Sets the icon for this dialog.

\wxheading{Parameters}

\docparam{icon}{The icon to associate with this dialog.}

See also \helpref{wxIcon}{wxicon}.


\membersection{wxDialog::SetIcons}\label{wxdialogseticons}

\func{void}{SetIcons}{\param{const wxIconBundle\& }{icons}}

Sets the icons for this dialog.

\wxheading{Parameters}

\docparam{icons}{The icons to associate with this dialog.}

See also \helpref{wxIconBundle}{wxiconbundle}.


\membersection{wxDialog::SetModal}\label{wxdialogsetmodal}

\func{void}{SetModal}{\param{const bool}{ flag}}

{\bf NB:} This function is deprecated and doesn't work for all ports, just use
\helpref{ShowModal}{wxdialogshowmodal} to show a modal dialog instead.

Allows the programmer to specify whether the dialog box is modal (wxDialog::Show blocks control
until the dialog is hidden) or modeless (control returns immediately).

\wxheading{Parameters}

\docparam{flag}{If true, the dialog will be modal, otherwise it will be modeless.}


\membersection{wxDialog::SetReturnCode}\label{wxdialogsetreturncode}

\func{void}{SetReturnCode}{\param{int }{retCode}}

Sets the return code for this window.

\wxheading{Parameters}

\docparam{retCode}{The integer return code, usually a control identifier.}

\wxheading{Remarks}

A return code is normally associated with a modal dialog, where \helpref{wxDialog::ShowModal}{wxdialogshowmodal} returns
a code to the application. The function \helpref{wxDialog::EndModal}{wxdialogendmodal} calls {\bf SetReturnCode}.

\wxheading{See also}

\helpref{wxDialog::GetReturnCode}{wxdialoggetreturncode}, \helpref{wxDialog::ShowModal}{wxdialogshowmodal},\rtfsp
\helpref{wxDialog::EndModal}{wxdialogendmodal}


\membersection{wxDialog::Show}\label{wxdialogshow}

\func{bool}{Show}{\param{const bool}{ show}}

Hides or shows the dialog.

\wxheading{Parameters}

\docparam{show}{If true, the dialog box is shown and brought to the front;
otherwise the box is hidden. If false and the dialog is
modal, control is returned to the calling program.}

\wxheading{Remarks}

The preferred way of dismissing a modal dialog is to use \helpref{wxDialog::EndModal}{wxdialogendmodal}.


\membersection{wxDialog::ShowModal}\label{wxdialogshowmodal}

\func{int}{ShowModal}{\void}

Shows a modal dialog. Program flow does not return until the dialog has been dismissed with\rtfsp
\helpref{wxDialog::EndModal}{wxdialogendmodal}.

\wxheading{Return value}

The return value is the value set with \helpref{wxDialog::SetReturnCode}{wxdialogsetreturncode}.

\wxheading{See also}

\helpref{wxDialog::EndModal}{wxdialogendmodal},\rtfsp
\helpref{wxDialog:GetReturnCode}{wxdialoggetreturncode},\rtfsp
\helpref{wxDialog::SetReturnCode}{wxdialogsetreturncode}


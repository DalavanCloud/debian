\section{\class{wxRichTextHTMLHandler}}\label{wxrichtexthtmlhandler}

Handles HTML output (only) for \helpref{wxRichTextCtrl}{wxrichtextctrl} content.

The most flexible way to use this class is to create a temporary object and call
its functions directly, rather than use \helpref{wxRichTextBuffer::SaveFile}{wxrichtextbuffersavefile} or\rtfsp
\helpref{wxRichTextCtrl::SaveFile}{wxrichtextctrlsavefile}.

Image handling requires a little extra work from the application, to choose an
appropriate image format for the target HTML viewer and to clean up the temporary images
later. If you are planning to load the HTML into a standard web browser, you can
specify the handler flag wxRICHTEXT\_HANDLER\_SAVE\_IMAGES\_TO\_BASE64 (the default)
and no extra work is required: the images will be written with the HTML.

However, if you want wxHTML compatibility, you will need to use wxRICHTEXT\_HANDLER\_SAVE\_IMAGES\_TO\_MEMORY
or wxRICHTEXT\_HANDLER\_SAVE\_IMAGES\_TO\_FILES. In this case, you must either call \helpref{DeleteTemporaryImages}{wxrichtexthtmlhandlerdeletetemporaryimages} before
the next load operation, or you must store the image
locations and delete them yourself when appropriate. You can call \helpref{GetTemporaryImageLocations}{wxrichtexthtmlhandlergettemporaryimagelocations} to
get the array of temporary image names.

\wxheading{Handler flags}

The following flags can be used with this handler, via
the handler's SetFlags function or the buffer or control's
SetHandlerFlags function:

\twocolwidtha{7cm}
\begin{twocollist}\itemsep=0pt
\twocolitem{\windowstyle{wxRICHTEXT\_HANDLER\_SAVE\_IMAGES\_TO\_MEMORY}}{Images are saved to the memory filesystem: suitable for showing wxHTML windows.}
\twocolitem{\windowstyle{wxRICHTEXT\_HANDLER\_SAVE\_IMAGES\_TO\_FILES}}{Images are saved to temporary files: suitable for showing in wxHTML windows.}
\twocolitem{\windowstyle{wxRICHTEXT\_HANDLER\_SAVE\_IMAGES\_TO\_BASE64}}{Images are written with the HTML files in Base 64 format: suitable for showing in web browsers.}
\twocolitem{\windowstyle{wxRICHTEXT\_HANDLER\_NO\_HEADER\_FOOTER}}{Don't include header and footer tags (HTML, HEAD, BODY), so that the HTML can be used as part of a larger document.}
\end{twocollist}

\wxheading{Derived from}

\helpref{wxRichTextFileHandler}{wxrichtextfilehandler}

\wxheading{Include files}

<wx/richtext/richtexthtml.h>

\wxheading{Data structures}

\latexignore{\rtfignore{\wxheading{Members}}}

\membersection{wxRichTextHTMLHandler::wxRichTextHTMLHandler}\label{wxrichtexthtmlhandlerwxrichtexthtmlhandler}

\func{}{wxRichTextHTMLHandler}{\param{const wxString\& }{name = wxT("HTML")}, \param{const wxString\& }{ext = wxT("html")}, \param{int }{type = wxRICHTEXT\_TYPE\_HTML}}

Constructor.

\membersection{wxRichTextHTMLHandler::ClearTemporaryImageLocations}\label{wxrichtexthtmlhandlercleartemporaryimagelocations}

\func{void}{ClearTemporaryImageLocations}{\void}

Clears the image locations generated by the last operation.

\membersection{wxRichTextHTMLHandler::DeleteTemporaryImages}\label{wxrichtexthtmlhandlerdeletetemporaryimages}

\func{bool}{DeleteTemporaryImages}{\void}

Deletes the in-memory or temporary files generated by the last operation.

\func{bool}{DeleteTemporaryImages}{\param{int }{flags}, \param{const wxArrayString\& }{imageLocations}}

Delete the in-memory or temporary files generated by the last operation. This is a static
function that can be used to delete the saved locations from an earlier operation,
for example after the user has viewed the HTML file.

\membersection{wxRichTextHTMLHandler::DoSaveFile}\label{wxrichtexthtmlhandlerdosavefile}

\func{bool}{DoSaveFile}{\param{wxRichTextBuffer* }{buffer}, \param{wxOutputStream\& }{stream}}

Saves the buffer content to the HTML stream.

\membersection{wxRichTextHTMLHandler::GetFontSizeMapping}\label{wxrichtexthtmlhandlergetfontsizemapping}

\func{wxArrayInt}{GetFontSizeMapping}{\void}

Returns the mapping for converting point sizes to HTML font sizes.

\membersection{wxRichTextHTMLHandler::GetTempDir}\label{wxrichtexthtmlhandlergettempdir}

\constfunc{const wxString\&}{GetTempDir}{\void}

Returns the directory used to store temporary image files.

\membersection{wxRichTextHTMLHandler::GetTemporaryImageLocations}\label{wxrichtexthtmlhandlergettemporaryimagelocations}

\constfunc{const wxArrayString\&}{GetTemporaryImageLocations}{\void}

Returns the image locations for the last operation.

\membersection{wxRichTextHTMLHandler::SetFileCounter}\label{wxrichtexthtmlhandlersetfilecounter}

\func{void}{SetFileCounter}{\param{int }{counter}}

Reset the file counter, in case, for example, the same names are required each time

\membersection{wxRichTextHTMLHandler::SetFontSizeMapping}\label{wxrichtexthtmlhandlersetfontsizemapping}

\func{void}{SetFontSizeMapping}{\param{const wxArrayInt\& }{fontSizeMapping}}

Sets the mapping for converting point sizes to HTML font sizes.
There should be 7 elements, one for each HTML font size, each element specifying the maximum point size for that
HTML font size.

For example:

\begin{verbatim}
    wxArrayInt fontSizeMapping;
    fontSizeMapping.Add(7);
    fontSizeMapping.Add(9);
    fontSizeMapping.Add(11);
    fontSizeMapping.Add(12);
    fontSizeMapping.Add(14);
    fontSizeMapping.Add(22);
    fontSizeMapping.Add(100);
    
    htmlHandler.SetFontSizeMapping(fontSizeMapping);
\end{verbatim}

\membersection{wxRichTextHTMLHandler::SetTempDir}\label{wxrichtexthtmlhandlersettempdir}

\func{void}{SetTempDir}{\param{const wxString\& }{tempDir}}

Sets the directory for storing temporary files. If empty, the system
temporary directory will be used.

\membersection{wxRichTextHTMLHandler::SetTemporaryImageLocations}\label{wxrichtexthtmlhandlersettemporaryimagelocations}

\func{void}{SetTemporaryImageLocations}{\param{const wxArrayString\& }{locations}}

Sets the list of image locations generated by the last operation.


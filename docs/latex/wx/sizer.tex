\section{\class{wxSizer}}\label{wxsizer}

wxSizer is the abstract base class used for laying out subwindows in a window. You
cannot use wxSizer directly; instead, you will have to use one of the sizer
classes derived from it. Currently there are \helpref{wxBoxSizer}{wxboxsizer}, 
\helpref{wxStaticBoxSizer}{wxstaticboxsizer},
\helpref{wxGridSizer}{wxgridsizer} 
\helpref{wxFlexGridSizer}{wxflexgridsizer} and \helpref{wxGridBagSizer}{wxgridbagsizer}.

The layout algorithm used by sizers in wxWidgets is closely related to layout
in other GUI toolkits, such as Java's AWT, the GTK toolkit or the Qt toolkit. It is
based upon the idea of the individual subwindows reporting their minimal required
size and their ability to get stretched if the size of the parent window has changed.
This will most often mean that the programmer does not set the original size of
a dialog in the beginning, rather the dialog will be assigned a sizer and this sizer
will be queried about the recommended size. The sizer in turn will query its
children, which can be normal windows, empty space or other sizers, so that
a hierarchy of sizers can be constructed. Note that wxSizer does not derive from wxWindow
and thus does not interfere with tab ordering and requires very little resources compared
to a real window on screen.

What makes sizers so well fitted for use in wxWidgets is the fact that every control
reports its own minimal size and the algorithm can handle differences in font sizes
or different window (dialog item) sizes on different platforms without problems. If e.g.
the standard font as well as the overall design of Motif widgets requires more space than
on Windows, the initial dialog size will automatically be bigger on Motif than on Windows.

Sizers may also be used to control the layout of custom drawn items on the window.  The
Add, Insert, and Prepend functions return a pointer to the newly added wxSizerItem. Just
add empty space of the desired size and attributes, and then use the wxSizerItem::GetRect
method to determine where the drawing operations should take place.


Please notice that sizers, like child windows, are owned by the library and
will be deleted by it which implies that they must be allocated on the heap.
However if you create a sizer and do not add it to another sizer or window, the
library wouldn't be able to delete such an orphan sizer and in this, and only
this, case it should be deleted explicitly.

\pythonnote{If you wish to create a sizer class in wxPython you should
derive the class from {\tt wxPySizer} in order to get Python-aware
capabilities for the various virtual methods.}

\wxheading{Derived from}

\helpref{wxObject}{wxobject}\\
\helpref{wxClientDataContainer}{wxclientdatacontainer}

\wxheading{Include files}

<wx/sizer.h>

\wxheading{See also}

\helpref{Sizer overview}{sizeroverview}

\latexignore{\rtfignore{\wxheading{Members}}}


\membersection{wxSizer::wxSizer}\label{wxsizerwxsizer}

\func{}{wxSizer}{\void}

The constructor. Note that wxSizer is an abstract base class and may not
be instantiated.


\membersection{wxSizer::\destruct{wxSizer}}\label{wxsizerdtor}

\func{}{\destruct{wxSizer}}{\void}

The destructor.


\membersection{wxSizer::Add}\label{wxsizeradd}

\func{wxSizerItem*}{Add}{\param{wxWindow* }{window}, \param{const wxSizerFlags\& }{flags}}

\func{wxSizerItem*}{Add}{\param{wxWindow* }{window}, \param{int }{proportion = 0},\param{int }{flag = 0}, \param{int }{border = 0}, \param{wxObject* }{userData = NULL}}

\func{wxSizerItem*}{Add}{\param{wxSizer* }{sizer}, \param{const wxSizerFlags\& }{flags}}

\func{wxSizerItem*}{Add}{\param{wxSizer* }{sizer}, \param{int }{proportion = 0}, \param{int }{flag = 0}, \param{int }{border = 0}, \param{wxObject* }{userData = NULL}}

\func{wxSizerItem*}{Add}{\param{int }{width}, \param{int }{height}, \param{int }{proportion = 0}, \param{int }{flag = 0}, \param{int }{border = 0}, \param{wxObject* }{userData = NULL}}

Appends a child to the sizer.  wxSizer itself is an abstract class, but the parameters are
equivalent in the derived classes that you will instantiate to use it so they are described
here:

\docparam{window}{The window to be added to the sizer. Its initial size (either set explicitly by the
user or calculated internally when using wxDefaultSize) is interpreted as the minimal and in many
cases also the initial size.}

\docparam{sizer}{The (child-)sizer to be added to the sizer. This allows placing a child sizer in a
sizer and thus to create hierarchies of sizers (typically a vertical box as the top sizer and several
horizontal boxes on the level beneath).}

\docparam{width and height}{The dimension of a spacer to be added to the sizer. Adding spacers to sizers
gives more flexibility in the design of dialogs; imagine for example a horizontal box with two buttons at the
bottom of a dialog: you might want to insert a space between the two buttons and make that space stretchable
using the \arg{proportion} flag and the result will be that the left button will be aligned with the left
side of the dialog and the right button with the right side - the space in between will shrink and grow with
the dialog.}

\docparam{proportion}{Although the meaning of this parameter is undefined in wxSizer, it is used in wxBoxSizer
to indicate if a child of a sizer can change its size in the main orientation of the wxBoxSizer - where
0 stands for not changeable and a value of more than zero is interpreted relative to the value of other
children of the same wxBoxSizer. For example, you might have a horizontal wxBoxSizer with three children, two
of which are supposed to change their size with the sizer. Then the two stretchable windows would get a
value of 1 each to make them grow and shrink equally with the sizer's horizontal dimension.}

\docparam{flag}{This parameter can be used to set a number of flags
which can be combined using the binary OR operator |. Two main
behaviours are defined using these flags. One is the border around a
window: the \arg{border} parameter determines the border width whereas
the flags given here determine which side(s) of the item that the
border will be added.  The other flags determine how the sizer item
behaves when the space allotted to the sizer changes, and is somewhat
dependent on the specific kind of sizer used.

\twocolwidtha{5cm}%
\begin{twocollist}\itemsep=0pt
\twocolitem{\windowstyle{wxTOP}\\
\windowstyle{wxBOTTOM}\\
\windowstyle{wxLEFT}\\
\windowstyle{wxRIGHT}\\
\windowstyle{wxALL}}{These flags are used to specify which side(s) of
  the sizer item the \arg{border} width will apply to. }

\twocolitem{\windowstyle{wxEXPAND}}{The item will be expanded to fill
the space assigned to the item.}
\twocolitem{\windowstyle{wxSHAPED}}{The item will be expanded as much
as possible while also maintaining its aspect ratio}
\twocolitem{\windowstyle{wxFIXED\_MINSIZE}}{Normally wxSizers will use 
\helpref{GetAdjustedBestSize}{wxwindowgetadjustedbestsize} to
determine what the minimal size of window items should be, and will
use that size to calculate the layout. This allows layouts to
adjust when an item changes and its \arg{best size} becomes
different. If you would rather have a window item stay the size it
started with then use wxFIXED\_MINSIZE.}
\twocolitem{\windowstyle{wxRESERVE\_SPACE\_EVEN\_IF\_HIDDEN}}{Normally wxSizers
don't allocate space for hidden windows or other items. This flag overrides
this behavior so that sufficient space is allocated for the window even if it
isn't visible. This makes it possible to dynamically show and hide controls
without resizing parent dialog, for example. \newsince{2.8.8}}
\twocolitem{\windowstyle{wxALIGN\_CENTER wxALIGN\_CENTRE}\\
\windowstyle{wxALIGN\_LEFT}\\
\windowstyle{wxALIGN\_RIGHT}\\
\windowstyle{wxALIGN\_TOP}\\
\windowstyle{wxALIGN\_BOTTOM}\\
\windowstyle{wxALIGN\_CENTER\_VERTICAL wxALIGN\_CENTRE\_VERTICAL}\\
\windowstyle{wxALIGN\_CENTER\_HORIZONTAL wxALIGN\_CENTRE\_HORIZONTAL}}{The wxALIGN flags allow you to
specify the alignment of the item within the space allotted to it by
the sizer, adjusted for the border if any.}
\end{twocollist}
}

\docparam{border}{Determines the border width, if the \arg{flag}
  parameter is set to include any border flag.}

\docparam{userData}{Allows an extra object to be attached to the sizer
item, for use in derived classes when sizing information is more
complex than the \arg{proportion} and \arg{flag} will allow for.}

\docparam{flags}{A \helpref{wxSizerFlags}{wxsizerflags} object that 
enables you to specify most of the above parameters more conveniently.}

\membersection{wxSizer::AddSpacer}\label{wxsizeraddspacer}

\func{wxSizerItem*}{AddSpacer}{\param{int }{size}}

Adds non-stretchable space to the sizer. More readable way of calling
\helpref{Add}{wxsizeradd}(size, size, 0).


\membersection{wxSizer::AddStretchSpacer}\label{wxsizeraddstretchspacer}

\func{wxSizerItem*}{AddStretchSpacer}{\param{int }{prop = 1}}

Adds stretchable space to the sizer. More readable way of calling
\helpref{Add}{wxsizeradd}(0, 0, prop).


\membersection{wxSizer::CalcMin}\label{wxsizercalcmin}

\func{wxSize}{CalcMin}{\void}

This method is abstract and has to be overwritten by any derived class.
Here, the sizer will do the actual calculation of its children minimal sizes.


\membersection{wxSizer::Clear}\label{wxsizerclear}

\func{void}{Clear}{\param{bool }{delete\_windows = false}}

Detaches all children from the sizer. If \arg{delete\_windows} is \true then child windows will also be deleted.


\membersection{wxSizer::ComputeFittingClientSize}\label{wxsizercomputefittingclientsize}

\func{wxSize}{ComputeFittingClientSize}{\param{wxWindow* }{window}}

Computes client area size for \arg{window} so that it matches the 
sizer's minimal size. Unlike \helpref{GetMinSize}{wxsizergetminsize}, this
method accounts for other constraints imposed on \arg{window}, namely display's
size (returned size will never be too large for the display) and maximum
window size if previously set by
\helpref{wxWindow::SetMaxSize}{wxwindowsetmaxsize}.

The returned value is suitable for passing to
\helpref{wxWindow::SetClientSize}{wxwindowsetclientsize}.

\newsince{2.8.8}

\wxheading{See also}

\helpref{ComputeFittingWindowSize}{wxsizercomputefittingwindowsize},
\helpref{Fit}{wxsizerfit}


\membersection{wxSizer::ComputeFittingWindowSize}\label{wxsizercomputefittingwindowsize}

\func{wxSize}{ComputeFittingWindowSize}{\param{wxWindow* }{window}}

Like \helpref{ComputeFittingClientSize}{wxsizercomputefittingclientsize},
but converts the result into \emph{window} size.

The returned value is suitable for passing to
\helpref{wxWindow::SetSize}{wxwindowsetsize} or
\helpref{wxWindow::SetMinSize}{wxwindowsetminsize}.

\newsince{2.8.8}

\wxheading{See also}

\helpref{ComputeFittingClientSize}{wxsizercomputefittingclientsize},
\helpref{Fit}{wxsizerfit}


\membersection{wxSizer::Detach}\label{wxsizerdetach}

\func{bool}{Detach}{\param{wxWindow* }{window}}

\func{bool}{Detach}{\param{wxSizer* }{sizer}}

\func{bool}{Detach}{\param{size\_t }{index}}

Detach a child from the sizer without destroying it. \arg{window} is the window to be
detached, \arg{sizer} is the equivalent sizer and \arg{index} is the position of
the child in the sizer, typically 0 for the first item. This method does not
cause any layout or resizing to take place, call \helpref{wxSizer::Layout}{wxsizerlayout}
to update the layout "on screen" after detaching a child from the sizer.

Returns true if the child item was found and detached, false otherwise.

\wxheading{See also}

\helpref{wxSizer::Remove}{wxsizerremove}


\membersection{wxSizer::Fit}\label{wxsizerfit}

\func{wxSize}{Fit}{\param{wxWindow* }{window}}

Tell the sizer to resize the \arg{window} to match the sizer's minimal size. This
is commonly done in the constructor of the window itself, see sample in the description
of \helpref{wxBoxSizer}{wxboxsizer}. Returns the new size.

For a top level window this is the total window size, not client size.

\wxheading{See also}

\helpref{ComputeFittingClientSize}{wxsizercomputefittingclientsize},
\helpref{ComputeFittingWindowSize}{wxsizercomputefittingwindowsize}


\membersection{wxSizer::FitInside}\label{wxsizerfitinside}

\func{void}{FitInside}{\param{wxWindow* }{window}}

Tell the sizer to resize the virtual size of the \arg{window} to match the sizer's
minimal size.  This will not alter the on screen size of the window, but may cause
the addition/removal/alteration of scrollbars required to view the virtual area in
windows which manage it.

\wxheading{See also}

\helpref{wxScrolledWindow::SetScrollbars}{wxscrolledwindowsetscrollbars},\rtfsp
\helpref{wxSizer::SetVirtualSizeHints}{wxsizersetvirtualsizehints}


\membersection{wxSizer::GetChildren}\label{wxsizergetchildren}

\func{wxSizerItemList\&}{GetChildren}{\void}

Returns the list of the items in this sizer. The elements of type-safe 
\helpref{wxList}{wxlist} \texttt{wxSizerItemList} are objects of type 
\helpref{wxSizerItem *}{wxsizeritem}.


\membersection{wxSizer::GetContainingWindow}\label{wxsizergetcontainingwindow}

\constfunc{wxWindow *}{GetContainingWindow}{\void}

Returns the window this sizer is used in or \NULL if none.


\membersection{wxSizer::GetItem}\label{wxsizergetitem}

\func{wxSizerItem *}{GetItem}{\param{wxWindow* }{window}, \param{bool }{recursive = false}}

\func{wxSizerItem *}{GetItem}{\param{wxSizer* }{sizer}, \param{bool }{recursive = false}}

\func{wxSizerItem *}{GetItem}{\param{size\_t }{index}}

Finds item of the sizer which holds given \arg{window}, \arg{sizer} or is located
in sizer at position \arg{index}.
Use parameter \arg{recursive} to search in subsizers too.

Returns pointer to item or NULL.


\membersection{wxSizer::GetSize}\label{wxsizergetsize}

\func{wxSize}{GetSize}{\void}

Returns the current size of the sizer.


\membersection{wxSizer::GetPosition}\label{wxsizergetposition}

\func{wxPoint}{GetPosition}{\void}

Returns the current position of the sizer.


\membersection{wxSizer::GetMinSize}\label{wxsizergetminsize}

\func{wxSize}{GetMinSize}{\void}

Returns the minimal size of the sizer. This is either the combined minimal
size of all the children and their borders or the minimal size set by 
\helpref{SetMinSize}{wxsizersetminsize}, depending on which is bigger.

Note that the returned value is \emph{client} size, not window size.  In
particular, if you use the value to set toplevel window's minimal or actual
size, you should convert it using
\helpref{wxWindow::ClientToWindowSize}{wxwindowclienttowindowsize} before
passing it to \helpref{wxWindow::SetMinSize}{wxwindowsetminsize} or
\helpref{wxWindow::SetSize}{wxwindowsetsize}.


\membersection{wxSizer::Hide}\label{wxsizerhide}

\func{bool}{Hide}{\param{wxWindow* }{window}, \param{bool }{recursive = false}}

\func{bool}{Hide}{\param{wxSizer* }{sizer}, \param{bool }{recursive = false}}

\func{bool}{Hide}{\param{size\_t }{index}}

Hides the \arg{window}, \arg{sizer}, or item at \arg{index}.
To make a sizer item disappear, use Hide() followed by \helpref{Layout()}{wxsizerlayout}.
Use parameter \arg{recursive} to hide elements found in subsizers.

Returns \true if the child item was found, \false otherwise.

\wxheading{See also}

\helpref{wxSizer::IsShown}{wxsizerisshown},\rtfsp
\helpref{wxSizer::Show}{wxsizershow}


\membersection{wxSizer::Insert}\label{wxsizerinsert}

\func{wxSizerItem*}{Insert}{\param{size\_t }{index}, \param{wxWindow* }{window}, \param{const wxSizerFlags\& }{flags}}

\func{wxSizerItem*}{Insert}{\param{size\_t }{index}, \param{wxWindow* }{window}, \param{int }{proportion = 0},\param{int }{flag = 0}, \param{int }{border = 0}, \param{wxObject* }{userData = NULL}}

\func{wxSizerItem*}{Insert}{\param{size\_t }{index}, \param{wxSizer* }{sizer}, \param{const wxSizerFlags\& }{flags}}

\func{wxSizerItem*}{Insert}{\param{size\_t }{index}, \param{wxSizer* }{sizer}, \param{int }{proportion = 0}, \param{int }{flag = 0}, \param{int }{border = 0}, \param{wxObject* }{userData = NULL}}

\func{wxSizerItem*}{Insert}{\param{size\_t }{index}, \param{int }{width}, \param{int }{height}, \param{int }{proportion = 0}, \param{int }{flag = 0}, \param{int }{border = 0}, \param{wxObject* }{userData = NULL}}

Insert a child into the sizer before any existing item at \arg{index}.

\docparam{index}{The position this child should assume in the sizer.}

See \helpref{wxSizer::Add}{wxsizeradd} for the meaning of the other parameters.


\membersection{wxSizer::InsertSpacer}\label{wxsizerinsertspacer}

\func{wxSizerItem*}{InsertSpacer}{\param{size\_t }{index}, \param{int }{size}}

Inserts non-stretchable space to the sizer. More readable way of calling
\helpref{Insert}{wxsizerinsert}(size, size, 0).


\membersection{wxSizer::InsertStretchSpacer}\label{wxsizerinsertstretchspacer}

\func{wxSizerItem*}{InsertStretchSpacer}{\param{size\_t }{index}, \param{int }{prop = 1}}

Inserts stretchable space to the sizer. More readable way of calling
\helpref{Insert}{wxsizerinsert}(0, 0, prop).


\membersection{wxSizer::IsShown}\label{wxsizerisshown}

\constfunc{bool}{IsShown}{\param{wxWindow* }{window}}

\constfunc{bool}{IsShown}{\param{wxSizer* }{sizer}}

\constfunc{bool}{IsShown}{\param{size\_t }{index}}

Returns \true if the \arg{window}, \arg{sizer}, or item at \arg{index} is shown.

\wxheading{See also}

\helpref{wxSizer::Hide}{wxsizerhide},\rtfsp
\helpref{wxSizer::Show}{wxsizershow}


\membersection{wxSizer::Layout}\label{wxsizerlayout}

\func{void}{Layout}{\void}

Call this to force layout of the children anew, e.g. after having added a child
to or removed a child (window, other sizer or space) from the sizer while keeping
the current dimension.


\membersection{wxSizer::Prepend}\label{wxsizerprepend}

\func{wxSizerItem*}{Prepend}{\param{wxWindow* }{window}, \param{const wxSizerFlags\& }{flags}}

\func{wxSizerItem*}{Prepend}{\param{wxWindow* }{window}, \param{int }{proportion = 0}, \param{int }{flag = 0}, \param{int }{border = 0}, \param{wxObject* }{userData = NULL}}

\func{wxSizerItem*}{Prepend}{\param{wxSizer* }{sizer}, \param{const wxSizerFlags\& }{flags}}

\func{wxSizerItem*}{Prepend}{\param{wxSizer* }{sizer}, \param{int }{proportion = 0}, \param{int }{flag = 0}, \param{int }{border = 0}, \param{wxObject* }{userData = NULL}}

\func{wxSizerItem*}{Prepend}{\param{int }{width}, \param{int }{height}, \param{int }{proportion = 0}, \param{int }{flag = 0}, \param{int }{border= 0}, \param{wxObject* }{userData = NULL}}

Same as \helpref{wxSizer::Add}{wxsizeradd}, but prepends the items to the beginning of the
list of items (windows, subsizers or spaces) owned by this sizer.


\membersection{wxSizer::PrependSpacer}\label{wxsizerprependspacer}

\func{wxSizerItem*}{PrependSpacer}{\param{int }{size}}

Prepends non-stretchable space to the sizer. More readable way of calling
\helpref{Prepend}{wxsizerprepend}(size, size, 0).


\membersection{wxSizer::PrependStretchSpacer}\label{wxsizerprependstretchspacer}

\func{wxSizerItem*}{PrependStretchSpacer}{\param{int }{prop = 1}}

Prepends stretchable space to the sizer. More readable way of calling
\helpref{Prepend}{wxsizerprepend}(0, 0, prop).


\membersection{wxSizer::RecalcSizes}\label{wxsizerrecalcsizes}

\func{void}{RecalcSizes}{\void}

This method is abstract and has to be overwritten by any derived class.
Here, the sizer will do the actual calculation of its children's positions
and sizes.


\membersection{wxSizer::Remove}\label{wxsizerremove}

\func{bool}{Remove}{\param{wxWindow* }{window}}

\func{bool}{Remove}{\param{wxSizer* }{sizer}}

\func{bool}{Remove}{\param{size\_t }{index}}

Removes a child from the sizer and destroys it if it is a sizer or a spacer,
but not if it is a window (because windows are owned by their parent window,
not the sizer).  \arg{sizer} is the wxSizer to be removed,
\arg{index} is the position of the child in the sizer, e.g. $0$ for the first item.
This method does not cause any layout or resizing to take place, call
\helpref{wxSizer::Layout}{wxsizerlayout} to update the layout "on screen" after removing a
child from the sizer.

{\bf NB:} The method taking a wxWindow* parameter is deprecated as it does not
destroy the window as would usually be expected from Remove.  You should use 
\helpref{wxSizer::Detach}{wxsizerdetach} in new code instead.  There is
currently no wxSizer method that will both detach and destroy a wxWindow item.

Returns true if the child item was found and removed, false otherwise.


\membersection{wxSizer::Replace}\label{wxsizerreplace}

\func{bool}{Replace}{\param{wxWindow* }{oldwin}, \param{wxWindow* }{newwin}, \param{bool }{recursive = false}}

\func{bool}{Replace}{\param{wxSizer* }{oldsz}, \param{wxSizer* }{newsz}, \param{bool }{recursive = false}}

\func{bool}{Remove}{\param{size\_t }{oldindex}, \param{wxSizerItem* }{newitem}}

Detaches the given \arg{oldwin}, \arg{oldsz} child from the sizer and 
replaces it with the given window, sizer, or wxSizerItem.

The detached child is removed {\bf only} if it is a sizer or a spacer
(because windows are owned by their parent window, not the sizer).

Use parameter \arg{recursive} to search the given element recursively in subsizers.


This method does not cause any layout or resizing to take place, call
\helpref{wxSizer::Layout}{wxsizerlayout} to update the layout "on screen" after replacing a
child from the sizer.

Returns true if the child item was found and removed, false otherwise.


\membersection{wxSizer::SetDimension}\label{wxsizersetdimension}

\func{void}{SetDimension}{\param{int }{x}, \param{int }{y}, \param{int }{width}, \param{int }{height}}

Call this to force the sizer to take the given dimension and thus force the items owned
by the sizer to resize themselves according to the rules defined by the parameter in the 
\helpref{Add}{wxsizeradd} and \helpref{Prepend}{wxsizerprepend} methods.


\membersection{wxSizer::SetMinSize}\label{wxsizersetminsize}

\func{void}{SetMinSize}{\param{int }{width}, \param{int }{height}}

\func{void}{SetMinSize}{\param{const wxSize\& }{size}}

Call this to give the sizer a minimal size. Normally, the sizer will calculate its
minimal size based purely on how much space its children need. After calling this
method \helpref{GetMinSize}{wxsizergetminsize} will return either the minimal size
as requested by its children or the minimal size set here, depending on which is
bigger.


\membersection{wxSizer::SetItemMinSize}\label{wxsizersetitemminsize}

\func{void}{SetItemMinSize}{\param{wxWindow* }{window}, \param{int}{ width}, \param{int}{ height}}

\func{void}{SetItemMinSize}{\param{wxSizer* }{sizer}, \param{int}{ width}, \param{int}{ height}}

\func{void}{SetItemMinSize}{\param{size\_t }{index}, \param{int}{ width}, \param{int}{ height}}

Set an item's minimum size by window, sizer, or position. The item will be found recursively
in the sizer's descendants. This function enables an application to set the size of an item
after initial creation.


\membersection{wxSizer::SetSizeHints}\label{wxsizersetsizehints}

\func{void}{SetSizeHints}{\param{wxWindow* }{window}}

This method first calls \helpref{wxSizer::Fit}{wxsizerfit} and then 
\helpref{SetSizeHints}{wxtoplevelwindowsetsizehints} on the {\it window}
passed to it. This only makes sense when {\it window} is actually a
\helpref{wxTopLevelWindow}{wxtoplevelwindow} such as a wxFrame or a
wxDialog, since SetSizeHints only has any effect in these classes. 
It does nothing in normal windows or controls.

This method is commonly invoked in the constructor of a toplevel window itself
(see the sample in the description of \helpref{wxBoxSizer}{wxboxsizer}) if the
toplevel window is resizable.

\membersection{wxSizer::SetVirtualSizeHints}\label{wxsizersetvirtualsizehints}

\func{void}{SetVirtualSizeHints}{\param{wxWindow* }{window}}

Tell the sizer to set the minimal size of the \arg{window} virtual area to match the sizer's
minimal size. For windows with managed scrollbars this will set them appropriately.

\wxheading{See also}

\helpref{wxScrolledWindow::SetScrollbars}{wxscrolledwindowsetscrollbars}


\membersection{wxSizer::Show}\label{wxsizershow}

\func{bool}{Show}{\param{wxWindow* }{window}, \param{bool }{show = true}, \param{bool }{recursive = false}}

\func{bool}{Show}{\param{wxSizer* }{sizer}, \param{bool }{show = true}, \param{bool }{recursive = false}}

\func{bool}{Show}{\param{size\_t }{index}, \param{bool }{show = true}}

Shows or hides the \arg{window}, \arg{sizer}, or item at \arg{index}.
To make a sizer item disappear or reappear, use Show() followed by \helpref{Layout()}{wxsizerlayout}.
Use parameter \arg{recursive} to show or hide elements found in subsizers.

Returns true if the child item was found, false otherwise.

\wxheading{See also}

\helpref{wxSizer::Hide}{wxsizerhide},\rtfsp
\helpref{wxSizer::IsShown}{wxsizerisshown}




\section{\class{wxSizerFlags}}\label{wxsizerflags}

Normally, when you add an item to a sizer via 
\helpref{wxSizer::Add}{wxsizeradd}, you have to specify a lot of flags and
parameters which can be unwieldy. This is where wxSizerFlags comes in: it
allows you to specify all parameters using the named methods instead. For
example, instead of

\begin{verbatim}
    sizer->Add(ctrl, 0, wxEXPAND | wxBORDER, 10);
\end{verbatim}

you can now write

\begin{verbatim}
    sizer->Add(ctrl, wxSizerFlags().Expand().Border(10));
\end{verbatim}

This is more readable and also allows you to create wxSizerFlags objects which
can be reused for several sizer items.
\begin{verbatim}
    wxSizerFlags flagsExpand(1);
    flagsExpand.Expand().Border(10);

    sizer->Add(ctrl1, flagsExpand);
    sizer->Add(ctrl2, flagsExpand);
\end{verbatim}

Note that by specification, all methods of wxSizerFlags return the wxSizerFlags
object itself to allowing chaining multiple methods calls like in the examples
above.


\membersection{wxSizerFlags::wxSizerFlags}\label{wxsizerflagsctor}

\func{}{wxSizerFlags}{\param{int }{proportion = 0}}

Creates the wxSizer with the proportion specified by \arg{proportion}.


\membersection{wxSizerFlags::Align}\label{wxsizerflagsalign}

\func{wxSizerFlags\& }{Align}{\param{int }{align = 0}}

Sets the alignment of this wxSizerFlags to \arg{align}.

Note that if this method is not called, the wxSizerFlags has no specified alignment.

\wxheading{See also}

\helpref{Left}{wxsizerflagsleft},\\
\helpref{Right}{wxsizerflagsright},\\
\helpref{Centre}{wxsizerflagscentre}


\membersection{wxSizerFlags::Border}\label{wxsizerflagsborder}

\func{wxSizerFlags\& }{Border}{\param{int }{direction}, \param{int }{borderinpixels}}

\func{wxSizerFlags\& }{Border}{\param{int }{direction = wxALL}}

Sets the wxSizerFlags to have a border of a number of pixels specified by
\arg{borderinpixels} with the directions specified by \arg{direction}.

In the overloaded version without \arg{borderinpixels} parameter, the border of
default size, as returned by \helpref{GetDefaultBorder}{wxsizerflagsgetdefaultborder},
is used.


\membersection{wxSizerFlags::Center}\label{wxsizerflagscenter}

\func{wxSizerFlags\& }{Center}{\void}

Sets the object of the wxSizerFlags to center itself in the area it is given.


\membersection{wxSizerFlags::Centre}\label{wxsizerflagscentre}

\func{wxSizerFlags\& }{Centre}{\void}

\helpref{wxSizerFlags::Center}{wxsizerflagscenter} for people with the other dialect of english.


\membersection{wxSizerFlags::DoubleBorder}\label{wxsizerflagsdoubleborder}

\func{wxSizerFlags\& }{DoubleBorder}{\param{int }{direction = wxALL}}

Sets the border in the given \arg{direction} having twice the default border
size.


\membersection{wxSizerFlags::DoubleHorzBorder}\label{wxsizerflagsdoublehorzborder}

\func{wxSizerFlags\& }{DoubleHorzBorder}{\void}

Sets the border in left and right directions having twice the default border
size.


\membersection{wxSizerFlags::Expand}\label{wxsizerflagsexpand}

\func{wxSizerFlags\& }{Expand}{\void}

Sets the object of the wxSizerFlags to expand to fill as much area as it can.


\membersection{wxSizerFlags::GetDefaultBorder}\label{wxsizerflagsgetdefaultborder}

\func{static int}{GetDefaultBorder}{\void}

Returns the border used by default in \helpref{Border}{wxsizerflagsborder} method.


\membersection{wxSizerFlags::Left}\label{wxsizerflagsleft}

\func{wxSizerFlags\& }{Left}{\void}

Aligns the object to the left, shortcut for \texttt{Align(wxALIGN\_LEFT)}

\wxheading{See also}

\helpref{Align}{wxsizerflagsalign}


\membersection{wxSizerFlags::FixedMinSize}\label{wxsizerflagsfixedminsize}

\func{wxSizerFlags\& }{FixedMinSize}{\void}

Set the \texttt{wxFIXED\_MINSIZE} flag which indicates that the initial size of
the window should be also set as its minimal size.


\membersection{wxSizerFlags::Proportion}\label{wxsizerflagsproportion}

\func{wxSizerFlags\& }{Proportion}{\param{int }{proportion = 0}}

Sets the proportion of this wxSizerFlags to \arg{proportion}


\membersection{wxSizerFlags::ReserveSpaceEvenIfHidden}\label{wxsizerflagsreservespaceevenifhidden}

\func{wxSizerFlags\& }{ReserveSpaceEvenIfHidden}{\void}

Set the \texttt{wxRESERVE\_SPACE\_EVEN\_IF\_HIDDEN} flag. Normally wxSizers
don't allocate space for hidden windows or other items. This flag overrides
this behavior so that sufficient space is allocated for the window even if it
isn't visible. This makes it possible to dynamically show and hide controls
without resizing parent dialog, for example. \newsince{2.8.8}


\membersection{wxSizerFlags::Right}\label{wxsizerflagsright}

\func{wxSizerFlags\& }{Right}{\void}

Aligns the object to the right, shortcut for \texttt{Align(wxALIGN\_RIGHT)}

\wxheading{See also}

\helpref{Align}{wxsizerflagsalign}


\membersection{wxSizerFlags::Shaped}\label{wxsizerflagsshaped}

\func{wxSizerFlags\& }{Shaped}{\void}

Set the \texttt{wx\_SHAPED} flag which indicates that the elements should
always keep the fixed width to height ratio equal to its original value.


\membersection{wxSizerFlags::TripleBorder}\label{wxsizerflagstriplebleborder}

\func{wxSizerFlags\& }{TripleBorder}{\param{int }{direction = wxALL}}

Sets the border in the given \arg{direction} having thrice the default border
size.



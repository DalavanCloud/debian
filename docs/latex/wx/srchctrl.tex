%%%%%%%%%%%%%%%%%%%%%%%%%%%% wxSearchCtrl %%%%%%%%%%%%%%%%%%%%%%%%%%%%%%%%

\section{\class{wxSearchCtrl}}\label{wxsearchctrl}

A search control is a composite control with a search button, a text 
control, and a cancel button.

\wxheading{Derived from}

\helpref{wxTextCtrl}{wxtextctrl}\\
streambuf\\
\helpref{wxControl}{wxcontrol}\\
\helpref{wxWindow}{wxwindow}\\
\helpref{wxEvtHandler}{wxevthandler}\\
\helpref{wxObject}{wxobject}

\wxheading{Include files}

<wx/srchctrl.h>

\wxheading{Window styles}

\twocolwidtha{5cm}
\begin{twocollist}\itemsep=0pt
\twocolitem{\windowstyle{wxTE\_PROCESS\_ENTER}}{The control will generate
the event wxEVT\_COMMAND\_TEXT\_ENTER (otherwise pressing Enter key
is either processed internally by the control or used for navigation between
dialog controls).}
\twocolitem{\windowstyle{wxTE\_PROCESS\_TAB}}{The control will receive
wxEVT\_CHAR events for TAB pressed - normally, TAB is used for passing to the
next control in a dialog instead. For the control created with this style,
you can still use Ctrl-Enter to pass to the next control from the keyboard.}
\twocolitem{\windowstyle{wxTE\_NOHIDESEL}}{By default, the Windows text control
doesn't show the selection when it doesn't have focus - use this style to force
it to always show it. It doesn't do anything under other platforms.}
\twocolitem{\windowstyle{wxTE\_LEFT}}{The text in the control will be left-justified (default).}
\twocolitem{\windowstyle{wxTE\_CENTRE}}{The text in the control will be centered (currently wxMSW and wxGTK2 only).}
\twocolitem{\windowstyle{wxTE\_RIGHT}}{The text in the control will be right-justified (currently wxMSW and wxGTK2 only).}
\twocolitem{\windowstyle{wxTE\_CAPITALIZE}}{On PocketPC and Smartphone, causes the first letter to be capitalized.}
\end{twocollist}

See also \helpref{window styles overview}{windowstyles} and \helpref{wxSearchCtrl::wxSearchCtrl}{wxsearchctrlctor}.

\wxheading{Event handling}

To process input from a search control, use these event handler macros to direct input to member
functions that take a \helpref{wxCommandEvent}{wxcommandevent} argument. To retrieve actual search
queries, use EVT\_TEXT and EVT\_TEXT\_ENTER events, just as you would with \helpref{wxTextCtrl}{wxtextctrl}.

\twocolwidtha{9cm}%
\begin{twocollist}\itemsep=0pt
\twocolitem{{\bf EVT\_SEARCHCTRL\_SEARCH\_BTN(id, func)}}{Respond to a wxEVT\_SEARCHCTRL\_SEARCH\_BTN event,
generated when the search button is clicked. Note that this does not initiate a search.}
\twocolitem{{\bf EVT\_SEARCHCTRL\_CANCEL\_BTN(id, func)}}{Respond to a wxEVT\_SEARCHCTRL\_CANCEL\_BTN event,
generated when the cancel button is clicked.}
\end{twocollist}%


\latexignore{\rtfignore{\wxheading{Members}}}


\membersection{wxSearchCtrl::wxSearchCtrl}\label{wxsearchctrlctor}

\func{}{wxSearchCtrl}{\void}

Default constructor.

\func{}{wxSearchCtrl}{\param{wxWindow* }{parent}, \param{wxWindowID}{ id},\rtfsp
\param{const wxString\& }{value = ``"}, \param{const wxPoint\& }{pos = wxDefaultPosition}, \param{const wxSize\& }{size = wxDefaultSize},\rtfsp
\param{long}{ style = 0}, \param{const wxValidator\& }{validator = wxDefaultValidator}, \param{const wxString\& }{name = wxSearchCtrlNameStr}}

Constructor, creating and showing a text control.

\wxheading{Parameters}

\docparam{parent}{Parent window. Should not be NULL.}

\docparam{id}{Control identifier. A value of -1 denotes a default value.}

\docparam{value}{Default text value.}

\docparam{pos}{Text control position.}

\docparam{size}{Text control size.}

\docparam{style}{Window style. See \helpref{wxSearchCtrl}{wxsearchctrl}.}

\docparam{validator}{Window validator.}

\docparam{name}{Window name.}

\wxheading{See also}

\helpref{wxTextCtrl::Create}{wxtextctrlcreate}, \helpref{wxValidator}{wxvalidator}


\membersection{wxSearchCtrl::\destruct{wxSearchCtrl}}\label{wxsearchctrldtor}

\func{}{\destruct{wxSearchCtrl}}{\void}

Destructor, destroying the search control.


\membersection{wxSearchCtrl::SetMenu}\label{wxsearchctrlsetmenu}

\func{virtual void}{SetMenu}{\param{wxMenu* }{ menu}}

Sets the search control's menu object.  If there is already a menu associated with
the search control it is deleted.


\wxheading{Parameters}

\docparam{menu}{Menu to attach to the search control.}

\membersection{wxSearchCtrl::GetMenu}\label{wxsearchctrlgetmenu}

\func{virtual wxMenu*}{GetMenu}{\void}

Returns a pointer to the search control's menu object or NULL if there is no 
menu attached.


\membersection{wxSearchCtrl::ShowSearchButton}\label{wxsearchctrlshowsearchbutton}

\func{virtual void}{ShowSearchButton}{\param{bool }{ show}}

Sets the search button visibility value on the search control.  
If there is a menu attached, the search button will be visible regardless of the search
button visibility value. 

This has no effect in Mac OS X v10.3


\membersection{wxSearchCtrl::IsSearchButtonVisible}\label{wxsearchctrlissearchbuttonvisible}

\func{virtual bool}{IsSearchButtonVisible}{\void}

Returns the search button visibility value.  
If there is a menu attached, the search button will be visible regardless of the search
button visibility value. 

This always returns false in Mac OS X v10.3


\membersection{wxSearchCtrl::ShowCancelButton}\label{wxsearchctrlshowcancelbutton}

\func{virtual void}{ShowCancelButton}{\param{bool }{ show}}

Shows or hides the cancel button.


\membersection{wxSearchCtrl::IsCancelButtonVisible}\label{wxsearchctrliscancelbuttonvisible}

\func{virtual bool}{IsCancelButtonVisible}{\void}

Indicates whether the cancel button is visible.


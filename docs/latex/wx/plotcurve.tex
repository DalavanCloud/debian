%
% automatically generated by HelpGen from
% plot.h at 11/Feb/00 18:00:57
%

\section{\class{wxPlotCurve}}\label{wxplotcurve}

The wxPlotCurve class represents a curve displayed in a \helpref{wxPlotWindow}{wxplotwindow}. It
is a virtual curve, i.e. is acts only as an interface, leaving it to the programmer to care for
how the values pairs are matched. wxPlotWindow and wxPlotCurve are designed to display large
amounts of data, i.e. most typically data measured by some sort of machine.

This class is abstract, i.e. you have to derive your own class and implement the pure
virtual functions (\helpref{GetStartX()}{wxplotcurvegetstartx}, \helpref{GetEndX()}{wxplotcurvegetendx}
and \helpref{GetY()}{wxplotcurvegety}).

\wxheading{Derived from}

\helpref{wxObject}{wxobject}

\latexignore{\rtfignore{\wxheading{Members}}}

\membersection{wxPlotCurve::wxPlotCurve}\label{wxplotcurvewxplotcurve}

\func{}{wxPlotCurve}{\param{int }{offsetY}, \param{double }{startY}, \param{double }{endY}}

Constructor assigning start values. See below for interpretation.

\membersection{wxPlotCurve::GetEndX}\label{wxplotcurvegetendx}

\func{wxInt32}{GetEndX}{\void}

Must be overridden. This function should return the index of the last value
of this curve, typically 99 if 100 values have been measured.

\membersection{wxPlotCurve::GetEndY}\label{wxplotcurvegetendy}

\func{double}{GetEndY}{\void}

See \helpref{SetStartY}{wxplotcurvesetendy}.

\membersection{wxPlotCurve::GetOffsetY}\label{wxplotcurvegetoffsety}

\func{int}{GetOffsetY}{\void}

Returns the vertical offset.
\membersection{wxPlotCurve::GetY}\label{wxplotcurvegety}

\func{double}{GetY}{\param{wxInt32 }{x}}

Must be overridden. This function will return the actual Y value corresponding
to the given X value. The x value is of an integer type because it is considered
to be an index in row of measured values.

\membersection{wxPlotCurve::GetStartX}\label{wxplotcurvegetstartx}

\func{wxInt32}{GetStartX}{\void}

Must be overridden. This function should return the index of the first value
of this curve, typically zero.

\membersection{wxPlotCurve::GetStartY}\label{wxplotcurvegetstarty}

\func{double}{GetStartY}{\void}

See \helpref{SetStartY}{wxplotcurvesetstarty}.

\membersection{wxPlotCurve::SetEndY}\label{wxplotcurvesetendy}

\func{void}{SetEndY}{\param{double }{endY}}

The value returned by this function tells the plot window what the highest values
in the curve will be so that a suitable scale can be found for the display. If
the Y values in this curve are in the range of -1.5 to 0.5, this function should
return 0.5 or maybe 1.0 for nicer aesthetics.

\membersection{wxPlotCurve::SetOffsetY}\label{wxplotcurvesetoffsety}

\func{void}{SetOffsetY}{\param{int }{offsetY}}

When displaying several curves in one window, it is often useful to assign
different offsets to the curves. You should call \helpref{wxPlotWindow::Move}{wxplotwindowmove} 
to set this value after you have added the curve to the window.

\membersection{wxPlotCurve::SetStartY}\label{wxplotcurvesetstarty}

\func{void}{SetStartY}{\param{double }{startY}}

The value returned by this function tells the plot window what the lowest values
in the curve will be so that a suitable scale can be found for the display. If
the Y values in this curve are in the range of -1.5 to 0.5, this function should
return -1.5 or maybe -2.0 for nicer aesthetics.


%%%%%%%%%%%%%%%%%%%%%%%%%%%%%%%%%%%%%%%%%%%%%%%%%%%%%%%%%%%%%%%%%%%%%%%%%%%%%%%
%% Name:        choicebk.tex
%% Purpose:     wxChoicebook documentation
%% Author:      Vadim Zeitlin
%% Modified by: Wlodzimierz ABX Skiba from listbook.tex
%% Created:     15.09.04
%% RCS-ID:      $Id: choicebk.tex 38698 2006-04-13 14:22:32Z JS $
%% Copyright:   (c) Vadim Zeitlin, Wlodzimierz Skiba
%% License:     wxWindows license
%%%%%%%%%%%%%%%%%%%%%%%%%%%%%%%%%%%%%%%%%%%%%%%%%%%%%%%%%%%%%%%%%%%%%%%%%%%%%%%

\section{\class{wxChoicebook}}\label{wxchoicebook}

wxChoicebook is a class similar to \helpref{wxNotebook}{wxnotebook} but which
uses a \helpref{wxChoice}{wxchoice} to show the labels instead of the
tabs.

There is no documentation for this class yet but its usage is
identical to wxNotebook (except for the features clearly related to tabs
only), so please refer to that class documentation for now. You can also
use the \helpref{notebook sample}{samplenotebook} to see wxChoicebook in action.

wxChoicebook allows the use of wxBookCtrl::GetControlSizer, allowing a program
to add other controls next to the choice control. This is particularly useful
when screen space is restricted, as it often is when wxChoicebook is being employed.

\wxheading{Derived from}

wxBookCtrlBase\\
\helpref{wxControl}{wxcontrol}\\
\helpref{wxWindow}{wxwindow}\\
\helpref{wxEvtHandler}{wxevthandler}\\
\helpref{wxObject}{wxobject}

\wxheading{Include files}

<wx/choicebk.h>

\wxheading{Window styles}

\twocolwidtha{5cm}
\begin{twocollist}\itemsep=0pt

\twocolitem{\windowstyle{wxCHB\_DEFAULT}}{Choose the default location for the
labels depending on the current platform (left everywhere except Mac where
it is top).}
\twocolitem{\windowstyle{wxCHB\_TOP}}{Place labels above the page area.}
\twocolitem{\windowstyle{wxCHB\_LEFT}}{Place labels on the left side.}
\twocolitem{\windowstyle{wxCHB\_RIGHT}}{Place labels on the right side.}
\twocolitem{\windowstyle{wxCHB\_BOTTOM}}{Place labels below the page area.}

\end{twocollist}

\wxheading{See also}

\helpref{wxBookCtrl}{wxbookctrloverview}, \helpref{wxNotebook}{wxnotebook}, \helpref{notebook sample}{samplenotebook}

